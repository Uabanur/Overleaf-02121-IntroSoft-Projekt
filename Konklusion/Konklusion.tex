% Magnus
\section{Konklusion}
\textbf{Afgrænsning:} Målet med dette projekt var at designe og implementere et computerspil baseret på det klassiske brætspil dam. Med udgangspunkt i reglerne fra Dansk Dam Forening designede og implementerede vi først et simpelt spil \textsc{SimpDam}, og udvidede derefter \textsc{SimpDam} med til \textsc{AvaDam} ved at tilføje en række features. De tilføjede features blev præsenteret i afsnit \ref{sec:afgraensning}.\\

\textbf{Design:} I afsnit \ref{sec:designMVC} redegjorde vi for, hvordan vores designmønster tog udgangspunkt i model-view-control. I afsnit \ref{sec:designSimpDam} gjorde vi rede for hvordan brætstørrelsen ændres i \textsc{SimpDam} gennem terminalen eller konsollen. I afsnit \ref{sec:designAvaDam} gennemgik vi hvordan brætstørrelse og aktiverede regler ændres i \textsc{AvaDam} gennem menuerne, hvordan brugervejledningen er implicit i spillet gennem visuelle indikatorer, og hvilke features vi havde indført for at gøre spillet underholdende: tilpasselige brik-billeder, AI og animationer af brikbevægelser. \\

\textbf{Implementering:} I afsnit \ref{sec:ImpOverordnet} dokumenterede vi hvordan \textsc{SimpDam} og \texttt{AvaDam} er implementeret med \textsc{JavaFX}, hvordan vi anvendte static fields og metoder, og hvordan navngivningen af fields og metoder er anvendt i vores kildekode. 
I afsnit \ref{sec:ImpSimp} og \ref{sec:ImpAva} gennemgik vi spillets anvendte klasser, hvordan brættet og brikkerne lagres og opdateres, og i detaljer hvordan de tilføjede features er implementeret i \textsc{AvaDam}.\\

\textbf{Evaluering:} I afsnit \ref{sec:evaOverordnet} gav vi et indblik i hvordan vi som gruppe strukturerede projektet internt, og hvordan vi fik erfaring eksternt ved at udveksle spørgsmål og teste andre gruppers spil. I afsnit \ref{sec:evaSimp} og \ref{sec:evaAva} sammenlignede vi \textsc{SimpDam}s og \textsc{AvaDam}s sikkerhed.  Endeligt præsenterede vi de tilføjelser og ændringer, vi ikke nåede i dette projekt. \\

\textbf{Afsluttende} kan vi konkludere, at selvom vi sagtens kunne have brugt en uge mere, er projektet  lykkedes: Vi  har designet og implementeret et simpelt dam brætspil. Igennem projektet er vi blevet bedre til at organisere, planlægge og udføre samarbejde i en projektgruppe, og anvende softwareredskaber som \texttt{Java} og \texttt{JavaFX}.
